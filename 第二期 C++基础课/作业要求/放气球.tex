\documentclass[UTF8]{ctexart}
\usepackage{geometry}
\geometry{a4paper,scale=0.8}
% \title{你好,world!}
\author{lx}
\date{\today}
\begin{document}
% \maketitle
\section{放气球}
    \subsection{题目描述}
        小明有一个盒子和$n$个气球。某天他突发奇想,哭着闹着要在盒子里放气球,而且他只想以他选的$n$个固定的点为气球的球心放置。
        他用一个神奇的装置来放气球,首先,他把$n$个气球放在这$n$个点上,然后他可以以他想要的顺序来使气球膨胀,但是他不能使气球停止膨胀,
        当且仅当气球碰到盒子的某一面或者之前放的气球时才会停止膨胀,另外,如果某个点已经被一个气球所包含,那么这个点上的气球将不能膨胀了。
        现在他想知道所有气球的体积之和的最大值。
    \subsection{输入}
        第一行三个整数$a$ $b$ $c$ $(a,b,c\le 1000)$(分别对应$x,y,z$轴)表示盒子的长宽高。
        接下来一行一个整数$n(n\le 6)$表示气球(放置点)个数。
        接下来$n$行,每行是三个整数$x_i$ $y_i$ $z_i$,表示这些点的位置
    \subsection{输出}
        一个实数,表示所有气球的体积之和的最大值。小数点后至少6位,且数字总长度要恰好16,若不足则补前导0。

\end{document}